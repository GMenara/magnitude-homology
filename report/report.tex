\documentclass{article}
\usepackage[utf8]{inputenc}
\usepackage{amsmath}
\usepackage{amsthm}
\usepackage{hyperref}

\newtheorem{theorem}{Theorem}
\newtheorem{proposition}[theorem]{Proposition}
\newtheorem{lemma}[theorem]{Lemma}
\newtheorem{definition}[theorem]{Definition}
\newtheorem{remark}[theorem]{Remark}
\newtheorem{example}[theorem]{Example}
\newtheorem{conjecture}[theorem]{Conjecture}

\newcommand{\boundary}{\partial_{k,\ell}}
\newcommand{\im}{\emph{im}}

\title{Magnitude Homology - Report}
%\author{Giulia }
%\date
\begin{document}
	
	\maketitle
	
	\section{Goals}
	\begin{enumerate}
		\item Develop a library to efficiently compute magnitude homology. To do this we construct a matrix representing the MH boundary operator.
		\item Use MH for network analysis. Our hope is that MH can give us information about the structure of a network and how a particular structure influences information flow.
	\end{enumerate}
	
	\section{Magnitude Homology framework}
	\begin{definition}
		Let $G=(V,E)$ be a simple graph. We define the \emph{$(k,\ell)$-magnitude chain}, $MC_{k,\ell}(G)$, as the free abelian group generated by the $(k+1)$-tuples of vertices of $G$ such that the path $(x_0,...,x_k)$ has length $\ell$. That is,
		$$
		MC_{k,\ell}=<(x_0,...,x_k): (x_0,...,x_k)\in V^{k+1}, x_i \neq x_{i+1}, l(x_0,...,x_k)=\ell>.
		$$
		The \emph{$\ell$-magnitude complex} $MC_{*,\ell}(G)$ is the direct sum over $k$ of all $(k,\ell)$-magnitude chains,
		$$
		MC_{*,\ell}(G) = \bigoplus_{k\geq 0} MC_{k,\ell}(G).
		$$
	\end{definition}
	
	\begin{definition}
		The boundary operator $\boundary: MC_{k,\ell}(G) \to MC_{k-1,\ell}(G)$ is defined as follows:
		$$
		\boundary(x_0,...,x_k) = \sum_{i=0}^k a_i (x_0,...,\hat{x_i},...,x_k), \text{ where }
		$$
		
		\begin{equation*}
			a_i=\begin{cases} (-1)^i, \text{ if } l(x_0,...,\hat{x_i},...,x_k)=\ell \\
				0, \text{ otherwise.}
			\end{cases}
		\end{equation*}
	\end{definition}
	
	\begin{definition}
		The \emph{$(k,\ell)$-magnitude homology group}, $MH_{k,\ell}(G)$, is defined as
		$$
		MH_{k,\ell}(G) = \frac{\ker \boundary}{\im \partial_{k+1,\ell}}
		$$
	\end{definition}
	
	\subsection{Operative definition}
	Notice that the information provided by $MC_{k,l}$ is extremely noisy. This is because the definition only asks that each vertex $x_i$ of the $k$-tuple is different from $x_{i+1}$, and this does not prevent going back and forth between the same two vertices. That is, a generator of $MC_{4,4}$ is for example $(x_0,x_1,x_0,x_1,x_0)$.
	
	We attempt to overcome this issue with a slight modification of the definition, in which we erase from $MC_{k,l}$ all paths revisiting a vertex.
	
	\begin{definition}
		We define the \emph{$(k,\ell)$-reduced magnitude chain}, $MC_{k,\ell}^{red}(G)$, as the free abelian group generated by the $(k+1)$-tuples of \emph{different} vertices of $G$ such that the path $(x_0,...,x_k)$ has length $\ell$. That is,
		$$
		MC_{k,\ell}^{red}=<(x_0,...,x_k): (x_0,...,x_k)\in V^{k+1}, x_i \neq x_j, l(x_0,...,x_k)=\ell>.
		$$
		The \emph{$\ell$-reduced magnitude complex} $MC_{*,\ell}^{red}(G)$ is the direct sum over $k$ of all $(k,\ell)$-reduced magnitude chains,
		$$
		MC_{*,\ell}^{red}(G) = \bigoplus_{k\geq 0} MC_{k,\ell}^{red}(G).
		$$
	\end{definition}
	
	Since from now on we will only rely on $MC_{k,\ell}^{red}(G)$, we will indicate (with an abuse of notation) the $(k,\ell)$-reduced magnitude chain as $MC_{k,\ell}(G)$.
	
	\section{Matrix representing $\boundary$}
	\textbf{MODIFY HERE AND DESCRIBE SPARSE MATRIX}
	
	The matrix $\Delta_{k,\ell}$ representing $\boundary$ is constructed using the following algorithm:
	\begin{enumerate}
		\item find the tuples generating $MC_{k,\ell}(G)$ and $MC_{k-1,\ell}(G)$
		\item initialize an all-zeros matrix of dimension $MC_{k-1,\ell}(G) \times MC_{k,\ell}(G)$
		\item for $t \in MC_{k,\ell}(G)$, if $\boundary(t)=t' \in MC_{k-1,\ell}(G)$ change the entry $(t',t)$ to $-1$.
	\end{enumerate}
	
	\subsection{Complexity}
	
	\subsection{Code}
	The code can be found in \href{https://github.com/GMenara/magnitude-homology-calculations}{this} repository. 
	
	\subsection{Experiments}
	This matrix was proven to be effective in the computation of magnitude Betti numbers $\beta_{k,\ell}$ using examples taken from the paper ``Categorifying the magnitude of a graph", Hepworth and Willerton (arXiv: 1505.04125v2).  
	
	\section{Interpretation of the rank of $MH_{k,l}$.}
	
	\subsection{Diagonal $MH_{k,k}$}
	As for the original definition given by Hepworth and Willerton, the ranks of the $(0,0)$ and $(1,1)$ MH groups of a graph $G=(V,E)$ represent the cardinality of $V$ and the cardinality of $V$, respectively. 
	\medskip
	
	$MH_{k,k}$ provides us with the precise number of $3$-cycles and $4$-cycles contained in a graph (modulo automorphisms). Indeed, consider the following facts:
	\begin{itemize}
		\item the dimension of the kernel of our matrix $\Delta_{k,k}$ is equal to the number of all-zero columns plus the number of columns that are "copies" of a previously written column.
		\item the number of all-zero columns is (modulo automorphisms) equal to the number of triangles contained in the graph. This is because if a $(k,k)$-tuple $(x_0,x_1,...,x_k)$ is sent to zero after removing a vertex $x_i$, it means that the shortest path between $x_{i-1}$ and $x_{i+1}$ has length smaller than $2$. So there exists and edge $(x_{i-1},x_{i+1})$ and equivalently a triangle $(x_{i-1},x_i,x_{i+1})$.
		\item the number of "repeated" columns indicates how many $(k,k)$-tuples $(x_0,...,x_i,...,x_k)$ are sent to the same $(k-1,k)$-tuple $(x_0,...,\hat{x_i},...,x_k)$, and this provides us with upper and lower bounds for the number (modulo automorphisms) of $4$-cycles (EXPAND).
	\end{itemize}
	
	Therefore, to obtain the number of $3$-cycles contained in our graph we need to divide the number of all-zero columns by $6$, i.e. by the cardinality of the automorphisms group of the triangle $D_3$.
	
	\begin{remark}
		We can link this value to the global and local clustering coefficients, and to the cycle ratio defined in [PUT REFERENCE HERE].
		In case of a directed graph (i.e. network) this value accounts for transitivity.
	\end{remark}
	
	To obtain the number of $4$-cycles we first divide the number of "repeated" columns by two, in order to disregard the orientation, and then we again divide by two, so that two $(k-1,k)$-paths are glued into the same $4$-cycle.
	
	\begin{example}
		PUT HERE EXAMPLE OF ICOSAHEDRAL GRAPH.
	\end{example}
	
	Although the $(k,k)$-magnitude homology groups $MH_{k,k}$ all provide the same information, we point out that the rank of $MK_{k,k}$ might be different from the rank of $MK_{k',k'}$ when $k \neq k'$, and in particular rank($MH_{k,k}$)$<$rank($MH_{k',k'}$) for $k<k'$. 
	Consider for example the case of the icosahedral graph presented above.
	Here we have rank($MH_{2,2}$)$=180$ and rank($MH_{3,3}$)$=252$, and this is because some cycles are counted more than once in $MH_{3,3}$.
	For example, the tuples $(11,7,9,10)$, $(4,10,9,7) \in MC_{3,3}$ are both sent to $0$ because of the triangle $(7,9,10)$.
	
	Given this and considering the (obvious) fact that $MH_{2,2}$ is much faster to compute, in the future analysis we will just make use of $MH_{2,2}$.
	
	\subsection{Second diagonal $MH_{k-1,k}$}
	We are not yet able to provide a general interpretation for all magnitude homology groups $MH_{k-1,k}$ on the second diagonal, but we believe that $MH_{2,3}$ contains information regarding the number of $4$-cliques, $5$-cliques and $6$-cliques in the graph.
	
	Consider the following chain
	\begin{equation*}
		\begin{matrix}
			... &\to & MC_{3,3}          &\to & MC_{2,3}                &\to & MC_{1,3} &\to & 0 \\
			&    & (x_0,x_1,x_2,x_3) &\to & (x_0,\hat{x_1},x_2,x_3) &\to & (x_0,\hat{x_1},\hat{x_2},x_3) & &
		\end{matrix}
	\end{equation*}
	
	and assume $x_3 \neq \hat{x_1}$.
	
	If $(x_0,\hat{x_1},x_2,x_3) \in \ker(\partial_{2,3})$ then one of the following is true:
	\begin{itemize}
		\item Two different tuples in $MC_{2,3}$ are sent to the same tuple in $MC_{1,3}$, which means there is either a $4$-cycle or a $6$-cycle.
		\item The considered tuple in $MC_{2,3}$ is sent to zero, which means there exists either a $4$-cycle or a $5$-cycle.
	\end{itemize}
	
	Now, when we quotient by the image of $\partial_{3,3}$ we are in fact disregarding the elements $(x_0,\hat{x_1},x_2,x_3) \in MC_{2,3}$ such that $\ell(x_0,\hat{x_1},x_2,x_3)=\ell(x_0,x_1,x_2,x_3)$. That is, we are disregarding the tuples that do now contain the triangle $(x_0,x_1,x_2)$, and that therefore cannot be part of a clique.
	
	Summarizing, $MH_{2,3}$ contains information about $4,5,6$-cycles that contain all triangles, and this means counting $4$-cliques and candidates $5,6$-cliques.
	
	\begin{remark}
		We point out that the hypothesis "$x_3 \neq \hat{x_1}$" is crucial to obtain this interpretation. 
		Indeed, without this assumption it could happen that $x_3 = \hat{x_1}$, which would mean "revisiting an edge" and adding a lot of noise to $MH_{2,3}$.
	\end{remark}
	
	We recall the definition of \emph{diagonal graph} introduced by Hepworth and Willerton in [ADD REF].
	\begin{definition}
		A graph $G$ is called diagonal if $MH_{k,l}(G) = 0$ whenever $k \neq l$.
	\end{definition}
	
	What we noticed until now suggest the following fact
	
	\begin{proposition}
		If a graph $G$ is diagonal, then it is clique-free.
	\end{proposition}
	
	\begin{proof}
		Suppose G is diagonal, then $MH_{k,l}(G)=0$ whenever $k \neq l$. 
		In particular $MH_{2,3}=0$, meaning the graph contains no $4$-clique, and therefore no bigger clique.
	\end{proof}
	
	\section{Problems to solve}
	
	\begin{enumerate}
		\item The information in the magnitude homology groups $MH_{2,2}$ and $MH_{2,3}$ is "not divided", meaning we are just given the dimension of the kernel without the distinction between all-zero columns and repeated columns.
		\item Add in the software the hypothesis "$x_3 \neq \hat{x_1}$". This is not trivial because the software doesn't really "see" $\hat{x_1}$, it just computes the length of the path supported by the tuple $(x_0,\hat{x_1},x_2,x_3)$.
	\end{enumerate}

\section{Ideas for the future}

\begin{enumerate}
	\item Network analysis:
	\begin{itemize}
		\item Construct a time series using active nodes of a network
		\item Detect small cycles and cliques using MH
		\item Detect persistent structures using PH
	\end{itemize}

	\item Prove that (see if.. but I think so) MH is a stable tool:
	\begin{itemize}
		\item Define an "interaction index" which should be a measure the tendency of a general vertex $v$ to interact with other vertices in the graph (maybe taking inspiration from connectivity index [ADD RED] and cycle index [ADD RED]). Maybe the global clustering coefficient can be used?
		\item Define a distance between two graphs $G$ and $G'$ using the interaction index
		\item See if a small variation in the interaction implies a small variation in MH
		\item Maybe define a "MH diagram" following the idea of persistence diagrams and do the above point for this MH diagram  
	\end{itemize}
	
\end{enumerate}
	
	
\end{document}